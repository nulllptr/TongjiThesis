\chapter{车辆分布式控制并行仿真}
本章基于车辆分布式控制策略训练、验证需求,设计了并行仿真架构,包括车辆控制算法、仿真控制器和仿真环境三部分,实现了算法和仿真的解耦和多仿真并行化,便于不同的算法、模型训练和标定需求、仿真场景、仿真软件等进行灵活组合,并行执行,提高了仿真效率。

\section{仿真需求}

由于车辆分布式控制策略需要车辆自主根据信息规划自身运动,不同的车辆可能选取不同的控制策略,控制算法的输出也较为多样,对仿真环境和仿真软件的控制精细度要求不同,因此,系统采用模块化设计,各模块彼此独立,采用标准通信协议进行通信,便于根据需求进行模块替换和组合。

在基于解析模型的控制策略中,常常存在许多参数需要进行标定和调整,这一过程需要较多的数据支持。而在基于强化学习的车辆分布式控制策略训练中,则需要大量的数据进行预训练,同时,模型需要与仿真环境进行交互,以进一步提升自身效果,这也是强化学习训练过程中的瓶颈之一。此外,为了验证控制策略的通用性,常常需要针对不同场景、不同需求进行大量的仿真验证。因此,系统需要支持多仿真并行化,以快速生成数据、提高模型迭代和测试的效率。

\section{结构设计}

系统的结构设计如图
% \ref{fig:parallel_sim}
所示,主要包括车辆控制算法、仿真控制器和仿真环境三部分。

